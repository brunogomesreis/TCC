%%
%% This is file `example.tex',
%% generated with the docstrip utility.
%%
%% The original source files were:
%%
%% coppe.dtx  (with options: `example')
%% 
%% This is a sample monograph which illustrates the use of `coppe' document
%% class and `coppe-unsrt' BibTeX style.
%% 
%% \CheckSum{1416}
%% \CharacterTable
%%  {Upper-case    \A\B\C\D\E\F\G\H\I\J\K\L\M\N\O\P\Q\R\S\T\U\V\W\X\Y\Z
%%   Lower-case    \a\b\c\d\e\f\g\h\i\j\k\l\m\n\o\p\q\r\s\t\u\v\w\x\y\z
%%   Digits        \0\1\2\3\4\5\6\7\8\9
%%   Exclamation   \!     Double quote  \"     Hash (number) \#
%%   Dollar        \$     Percent       \%     Ampersand     \&
%%   Acute accent  \'     Left paren    \(     Right paren   \)
%%   Asterisk      \*     Plus          \+     Comma         \,
%%   Minus         \-     Point         \.     Solidus       \/
%%   Colon         \:     Semicolon     \;     Less than     \<
%%   Equals        \=     Greater than  \>     Question mark \?
%%   Commercial at \@     Left bracket  \[     Backslash     \\
%%   Right bracket \]     Circumflex    \^     Underscore    \_
%%   Grave accent  \`     Left brace    \{     Vertical bar  \|
%%   Right brace   \}     Tilde         \~}
%%
\documentclass[grad,numbers]{coppe}
\usepackage{amsmath,amssymb}
\usepackage{hyperref}
\usepackage[utf8]{inputenc}
\usepackage[brazil]{babel}
\usepackage[T1]{fontenc}
\usepackage{graphicx}
\usepackage{placeins}
\usepackage{textgreek}
\usepackage{amsmath}
\usepackage{amssymb}
\usepackage{enumitem}

\setcounter{secnumdepth}{3}

\makelosymbols
\makeloabbreviations

\begin{document}
  \title{Desenvolvimento de um sistema de baixo custo para medição de parâmetros da frequência respiratória com aplicação em neurociência comportamental humana.}
  \foreigntitle{low cost system designed to measure respiratory frequency parameters applied to human behavioral neuroscience}
  \author{Bruno Gomes}{Reis}
  \advisor{Prof.}{Carlos José Ribas}{D'Avila}{M.Sc.}
  \advisor{Prof.}{Tiago Arruda}{Sanchez}{Ph.D.}
  %\advisor{Prof.}{Nome do Terceiro Orientador}{Sobrenome}{D.Sc.}

  \examiner{Prof.}{Nome do Primeiro Examinador Sobrenome}{D.Sc.}
  \examiner{Prof.}{Nome do Segundo Examinador Sobrenome}{Ph.D.}
  \examiner{Prof.}{Nome do Terceiro Examinador Sobrenome}{D.Sc.}
  \examiner{Prof.}{Nome do Quarto Examinador Sobrenome}{Ph.D.}
  \examiner{Prof.}{Nome do Quinto Examinador Sobrenome}{Ph.D.}
  
  
  
  \department{EEC}% Confira a tabela a seguir para saber como preencher o comando \department de acordo com seu curso (Graduação - Poli) ou programa (Pós-Graduação - COPPE).
  
  %%%%%% Para alunos da POLI %%%%%%
  
  %% Course											Option
  %% Engenharia Ambiental                             EA
  %% Engenharia Civil                                 ECV
  %% Engenharia de Computação e Informação            ECI
  %% Engenharia de Controle e Automação               ECA
  %% Engenharia de Materiais                          EMAT
  %% Engenharia de Petróleo                           EPT
  %% Engenharia de Produção                           EPR
  %% Engenharia Eletrônica e de Computação            EEC
  %% Engenharia Elétrica                              EET
  %% Engenharia Mecânica                              EMC
  %% Engenharia Metalúrgica                           EMET
  %% Engenharia Naval e Oceânica                      ENO
  %% Engenharia Nuclear                               ENU
  
  
  %%%%%% Para alunos da COPPE %%%%%%
  
  %% Program											Option
  %% Engenharia Biomédica								PEB
  %% Engenharia Civil									PEC
  %% Engenharia Elétrica								PEE
  %% Engenharia Mecânica								PEM
  %% Engenharia Metalúrgica e de Materiais				PEMM
  %% Engenharia Nuclear									PEN
  %% Engenharia Oceânica								PENO
  %% Planejamento Energético							PPE
  %% Engenharia de Produção								PEP
  %% Engenharia Química									PEQ
  %% Engenharia de Sistemas e Computação				PESC
  %% Engenharia de Transportes							PET
  
  
  
  
  
  
  \date{03}{2020}

  \keyword{Behavioral Neuroscience}
  \keyword{Respiratory Frequency }
  \keyword{Engenharia Eletrônica e de Computação}

  \maketitle

  \frontmatter
  
  \makecatalog
  
  \dedication{Dedico este trabalho à ciência e à evolução da sociedade}

  \chapter*{Agradecimentos}

  Gostaria de agradecer a todos.

  \begin{abstract}
  	
  	O corpo humano é um sistema complexo cujo funcionamento e interações com o meio ambiente vem sendo estudado pela ciência contemporânea em constante evolução. O processo respiratório é composto basicamente de quatro etapas, sendo elas ventilação pulmonar, trocas gasosas, transporte sanguíneo e respiração celular. Pode-se observar que o sistema respiratório está intimamente ligado ao cardiovascular, uma vez que esse é o responsável pela circulação de sangue nas veias, ou seja, alterações na respiração podem ocasionar variações em outros aspectos da fisiologia humana, como a frequência cardíaca. Estudos recentes demonstram correlações entre diversas patologias e alterações comportamentais com mudanças nos padrões amplitude e frequência de inspiração e expiração no processo de ventilação. Motivado por estudos de neurociência comportamental, realizados no Laboratório de Neuroimagem e Psicofisiologia, surgiu a necessidade deste trabalho. Trata-se em sua essência, do desenvolvimento de um sistema de baixo custo capaz de mensurar, processar, exibir e exportar as informações de alguns parâmetros da ventilação respiratória, para que pesquisadores possam realizar análises de seus padrões em diversos experimentos.
  	
  	Este trabalho apresenta o desenvolvimento de um sistema composto por um conjunto de hardware e software, utilizando um termistor como sensor de fluxo para detectar a passagem de ar por uma máscara, um sistema de controle para garantir o autoaquecimento do sensor, um retificador de tensão capaz de entregar energia suficiente ao sistema, um microcontrolador que realiza a medição do sensor, a digitalização do resultado e a comunicação com o computador, um software de leitura e um sistema web capaz de realizar operações matemáticas, como análises no domínio da frequência e aplicação de filtros digitais passa-baixas, armazenamento de dados e exibição dos gráficos de forma interativa.
  	
  	
  	
  	
	    
  \end{abstract}

  \begin{foreignabstract}

  In this work, we present ...

  \end{foreignabstract}

  \tableofcontents
  \listoffigures
  \listoftables
  \printlosymbols
  \printloabbreviations

  \mainmatter
%  \doublespacing
  
    \chapter{Introdução}
  
  \section{Motivação}
  %Escrever sobre o cenário de software propício para chatbots. Falar sobre chatbots vs apps e uso das redes sociais.
  As redes sociais tem o papel de conectar pessoas, estreitando assim os laços humanos. Graças a elas podemos nos comunicar com pessoas que estão do outro lado do globo em tempo real. O sucesso das redes sociais é tão grande, que apenas no Brasil, temos cerca de 130 milhões de usuários mensais só no Facebook. Segundo dados da pesquisa 143ª Pesquisa CNT/MDA\footnote{https://www.cnt.org.br/agencia-cnt/confira-resultados-pesquisa-cnt-mda}, 82\% dos entrevistados afirmam que fazem uso de seus \emph{smartphones} para acesso à redes sociais. Além disso, os resultados da 29ª Pesquisa Anual de Administração e Uso de Tecnologia da Informação nas Empresas, realizada pela FGV, mostra que o Brasil já ultrapassou a marca de 1 aparelho \emph{smartphones} por habitante\footnote{https://eaesp.fgv.br/sites/eaesp.fgv.br/files/pesti2018gvciappt.pdf}.
  
  O tamanho sucesso das redes sociais e dos \emph{smartphones}, permitiu que os antigos sistemas de informação, que, em algumas décadas atrás necessitavam de grandes computadores para serem executados, agora são acessíveis por um dispositivo de mão. Jogos, bancos, redes sociais, e até mesmo redes de \emph{fast-food} desenvolveram seus aplicativos e marcaram presença nos \emph{smartphones} da população. Estima-se que apenas na Play Store (loja de aplicativos do Google), cerca de 2,6 milhões de aplicativos estejam disponíveis para \emph{download}\footnote{https://www.statista.com/statistics/266210/number-of-available-applications-in-the-google-play-store/}.
  
  Contudo, estudos mais recentes indicam que apenas cinco aplicativos são responsáveis por cerca de 84\% do uso de de aplicativos não nativos do smartphone\footnote{https://techcrunch.com/2015/06/22/consumers-spend-85-of-time-on-smartphones-in-apps-but-only-5-apps-see-heavy-use/}. Esses aplicativos variam de pessoa para pessoa, mas em geral, incluem aplicativos de redes sociais ou de mensagens instantâneas. Esse cenário justifica o aumento no número de chatbots disponíveis em plataformas de mensagens como Messenger nos últimos anos.
  
  % definir o que é chatbot %
  \textit{Chatbots} são sistemas computacionais que simulam uma conversas com pessoas e tornam a interação homem-máquina mais natural\cite{chatbot_definition}. São comumente chamados de assistentes virtuais, agentes virtuais ou simplesmente \textit{bots}.
  % / definir o que é chatbot %
  
  Ao criar um \emph{chatbot} para ser utilizado em canais já consolidados como Messenger, Telegram, Slack e WhatsApp, estamos utilizando uma infraestrutura já consolidada de um aplicativo para executar tarefas que um usuário faria em um outro aplicativo, o que dispensa a instalação e ocupação da memória do aparelho do usuário. Além disso, \textit{chatbots} também podem ser incluídos em \textit{sites} oferecendo aos visitantes um canal rápido de atendimento enquanto navega na internet.
  
  Outro ponto é que o formato de comunicação através de troca de mensagens oferecido pelos \emph{chatbots} é mais humano do que a interação em um aplicativo convencional. O usuário tem uma dúvida e, ao invés de navegar entre menus e telas de um aplicativo, ele simplesmente envia seu questionamento e é prontamente respondido pelo \textit{bot}.
  
  Do ponto de vista de quem oferece a aplicação, ter um \emph{chatbot} ao invés de um aplicativo significa não ter que competir para captar usuários em uma loja de aplicativos, que por sua vez é repleta de outros aplicativos que são potenciais concorrentes.  Além disso, o custo de desenvolvimento tende a ser inferior, o que também é uma vantagem. Ao criar um \emph{chatbot}, criamos um único serviço que poderá estar disponível em um ou mais canais de comunicação, utilizando a mesma estrutura. Quando criamos aplicativos, especialmente aqueles que estarão disponíveis em várias plataformas, gastamos tempo pensando na experiência do usuário para dois ou mais sistemas diferentes e, por fim desenvolvendo dois aplicativos com estruturas diferentes.
  
  
  
  
  % https://br.newsroom.fb.com/company-info/
  
  % https://www1.folha.uol.com.br/tec/2018/07/facebook-chega-a-127-milhoes-de-usuarios-mensais-no-brasil.shtml
  
  % https://exame.abril.com.br/negocios/dino/62-da-populacao-brasileira-esta-ativa-nas-redes-sociais/
  
  % https://www.cnt.org.br/agencia-cnt/confira-resultados-pesquisa-cnt-mda
  
  % https://eaesp.fgv.br/sites/eaesp.fgv.br/files/pesti2018gvciappt.pdf
  
  % https://techcrunch.com/2015/06/22/consumers-spend-85-of-time-on-smartphones-in-apps-but-only-5-apps-see-heavy-use/
  
  % Avaliar um chatbot --------- https://www.forbes.com/sites/forbesagencycouncil/2018/06/04/using-facebook-messenger-and-chatbots-to-grow-your-audience/#760a4a8733b1
  
  \section{Objetivos}
  % O que se espera ter como produto final?
  % - O objetivo é construir o chatbot tendo como base boas práticas de ES.
  % Que portal? Você não disse que teremos duas versões do bot, etc. 
  
  O objetivo principal desse trabalho é construir um \textit{chatbot} tendo como base boas práticas de engenharia de software. O \textit{bot} deverá ser disponibilizado em duas plataformas: \textit{web}, para ser incorporado ao Portal do PIPA (site institucional do projeto, onde os pacientes cadastrados podem consultar informações e exames) e \textit{Messenger}. No que diz respeito a funcionalidades, ele deverá ser capaz de dar informações sobre o projeto para o público geral e poderá acessar os usuários cadastrados no portal PIPA, podendo vincular um paciente ao usuário do \textit{bot}. Aos pacientes que realizaram esse vínculo, será permitido obter informações de consultas e resultados de exames através do \textit{chatbot}.
  
  Para que o \emph{chatbot} acesse a base de pacientes do PIPA, deverá ser criado também um novo \emph{endpoint} a API já existente no Portal. Esse \emph{endpoint} será protegido com uma tecnologia de \textit{token} para garantir que apenas solicitações autorizadas tenham acesso a dados do projeto.
  
  
  
  \section{Metodologia}
  % Falar sobre Lean Inception
  %TAÍSA: falar um pouco mais. Colocar referencia, talvez colocar uma figura.
  A metodologia utilizada para o desenvolvimento desse projeto segue a linha "Construir - Medir - Aprender" herdada do \emph{Lean Inception}, cuja finalidade é construir versões incrementais do Mínimo Produto Viável (MVP) a cada ciclo.
  
  \begin{figure}[h!]
  	\begin{center}
  		\includegraphics[width=10cm]{images/lean.png}
  		\caption{Lean Startup}
  	\end{center}
  \end{figure}

  Cada ciclo parte de uma ideia na qual o MVP é concebido (CONSTRUIR). Esse produto é colocado no mundo real, para ser validado pelos \textit{stakeholders} (MEDIR). Com o resultado dessas avaliações é necessário identificar, de acordo com a situação, se a próxima etapa será um incremento, pivotamento ou uma nova concepção do zero (APRENDER).
  
  \section{Organização do Trabalho}
  Nos próximos capítulos serão apresentados conceitos, ferramentas e os resultados obtidos com esse trabalho, organizados da seguinte forma:
  
  O capítulo 2 apresentará uma revisão da literatura, trazendo diversos conceitos a respeito de \emph{chatbots}, como seus tipos e também possíveis usos.
  
  O capítulo 3 enumerará uma série de ferramentas, \emph{frameworks} e canais que podem ser utilizados para se construir \emph{chatbots}.
  
  O capítulo 4 apresentará o contexto do projeto PIPA Bot, bem como as escolhas do projeto e suas devidas justificativas, ambientes de desenvolvimento, arquitetura do \emph{chatbot} e as versões do MVP construídas.
  
  O capítulo 5 apresentará a conclusão desse projeto, com o MVP construído no último ciclo de desenvolvimento, limitações encontradas ao longo das etapas e também possíveis melhorias para trabalhos futuros.
 

  \chapter{Motivação}
  
\section{Fisiologia da Respiração} \label{sec:fisiologiadarespiracao}

	A respiração possibilita trocas gasosas em vista da produção de energia pela absorção de oxigênio e eliminação do gás carbônico. Ela é, em resumo, um ato semiautomático no qual um controle involuntário é exercido por centros respiratórios no tronco encefálico e um controle voluntário é exercido por centros motores de músculos acessórios da respiração no córtex motor. O processo respiratório compreende quatro processos \cite{porto2014} são eles:
	
	\begin{itemize}
		\item [1-] Ventilação Pulmonar: Com o objetivo de levar o ar aos alvéolos, distribuindo-o adequadamente.
		\item [2-] Trocas Gasosas: Devida a diferença de pressão parcial dos gases $O_2$ e $CO_2$ nos alvéolos e no sangue, ocorre a passagem dos mesmos através da membrana alveolocapilar.
		\item [3-] Transporte Sanguíneo dos Gases: Tanto na etapa anterior quanto nesta etapa, é importante a interação dos processos respiratórios com o sistema circulatório. A circulação sistêmica promove a distribuição periférica do oxigênio e a extração de $CO_2$, havendo a captação de $O_2$ pela hemoglobina.
		\item [4-] Respiração Celular: Etapa terminal do processo e sua finalidade maior. Por meio da respiração celular, consubstancia-se a utilização celular
		do oxigênio por meio das cadeias enzimáticas mitocondriais.
	\end{itemize}
	
	A ventilação ocorre através da ação de músculos respiratórios que contraem-se de maneira coordenada aumentando e reduzindo o volume da cavidade torácica. A inspiração possui como músculo principal o diafragma, que se contrai, desce e expande a cavidade torácica, comprimindo então o conteúdo abdominal e empurrando para fora a parede abdominal. Concorrentemente, os músculos da caixa torácica também expandem o tórax, em especial, os músculos escalenos que percorrem das vértebras cervicais às duas primeiras costelas e os músculos intercostais paraesternais, que possuem um trajeto desde o externo até as costelas. Na medida em que o tórax se expande, a pressão intratorácica diminui, deslocando o ar da árvore traqueobrônquica para os alvéolos, preenchendo os pulmões em expansão. O oxigênio é então difundido para os capilares pulmonares adjacentes, enquanto o dióxido de carbono sai do sangue para os alvéolos. A expiração ocorre de maneira passiva, em decorrência da retração elástica dos pulmões e pelo relaxamento dos músculos respiratórios.
	
	Em estado normal, a respiração é tranquila, audível próximo à boca, sendo possível observar apenas os movimentos abdominais com facilidade. Contudo, durante a prática de atividade física, ou em decorrência de determinadas doenças, faz-se necessário um esforço respiratório adicional, recrutando músculos acessórios e demandando ainda mais esforço dos músculos abdominais, que passa a auxiliar também na expiração.
	
\section{Padrões Respiratórios} \label{sec:padroesrespiratorios}

	A faixa de frequência padrão para adultos normais é de, aproximadamente, 14 à 20 incursões por minuto e até 44 em lactantes. Em sua normalidade, a inspiração e expiração possuem o mesmo tempo e amplitude, sendo intercalados por uma leve pausa. No momento em que uma dessas características é modificada, surgem os ritmos respiratórios anormais, tais quais a respiração de Cheyne-Stokes, respiração de Biot, respiração de Kussmaul e respiração suspirosa.
	
	Respiração de Cheyne-Stokes: Frequentemente causada por insuficiência cardíaca, hipertensão intracraniana, acidentes vasculares cerebrais e traumatismos cranioencefálicos. Caracteriza-se por uma fase de apneia seguida de incursões inspiratórias cada vez mais profundas até atingir um máximo e, em seguida, decrescer até uma nova pausa. Esse comportamento ocorre devido a variações da tensão de $O_2$ e $CO_2$ no sangue. O excesso de $CO_2$ durante o período de apneia obriga os centros respiratórios a enviar estímulos mais intensos, resultando em um aumento da amplitude dos movimentos respiratórios, em consequência, haverá uma maior eliminação de $CO_2$ fazendo com que a concentração deste no sangue decaia. Após essa queda, os centros respiratórios recebem o comando inverso, enviando estímulos menores e diminuindo a amplitude dos movimentos respiratórios até que aumente novamente a concentração de $CO_2$ no sangue e o ciclo se repita.
	
	Respiração de Biot (ou atáxica): As causas dessa respiração são similares às da respiração de Cheyne-Stokes, porém, no ritmo de Biot, a respiração possui duas etapas, uma de apneia seguida de outra com movimentos respiratórios anárquicos quanto ao ritmo e à amplitude. Esse padrão respiratório quase sempre indica um grave comprometimento cerebral.
	
	Respiração de Kussmaul: Costuma ser causada pela acidose (diminuição do PH sanguíneo para menos de 7,35[aumento de H+]), principalmente a diabética. O padrão respiratório é composto de quatro fases: Inspirações ruidosas, gradativamente mais amplas, alternada com inspirações rápidas e de pequena amplitude; Apneia em inspiração; Expirações ruidosas, gradativamente mais profundas, alternadas com inspirações rápidas e de pequena amplitude;	
	E apneia em expiração.
	
	Respiração suspirosa: Normalmente, pode ser traduzida em tensão emocional e ansiedade. Seu padrão ocorre quando o paciente executa uma série de movimentos inspiratórios de amplitude crescente, seguidos de uma expiração breve e rápida.
	
	Além desses padrões respiratórios, existem diversos outros padrões observados (como a Taqpineia, Bradpneia, Respiração obstrutiva e Hiperpneia) que possuem correlação com patologias ou alterações emocionais e, portanto, existe um vasto campo de estudo acerca desse tema.
	
\section{Aplicação do sistema}

	Esse trabalho é fruto da cooperação entre a Faculdade de Medicina e a Engenharia Eletrônica, representadas pelos professores Tiago Arruda Sanchez e Carlos José Ribas D'Avila respectivamente. O professor Tiago desenvolve estudos na área de neurociência comportamental no laboratório de neuroimagem e psicofisiologia, situado no Hospital Universitário Clementino Fraga Filho. Com a principal finalidade de auxiliar em estudos nas áreas de interesse do laboratório, o desenvolvimento do sistema visa trazer mais insumos às análises realizadas nos diversos projetos existentes no laboratório. Conforme mencionado na sessão \ref{sec:padroesrespiratorios}, existem diferentes padrões respiratórios já observados entre os quais se é possível obter uma relação de correlação com patologias e alterações emocionais e comportamentais. Por derradeiro, a existência de um sistema capaz de obter dados referentes ao comportamento respiratório, processar e exibir seus dados transcreve-se em um ganho à neurociência comportamental pelo qual este projeto se justifica.
  

  \chapter{Desenvolvimento}
  
\section{O funcionamento do termistor}
 %Motivação sobre o uso do termistor (menos invasivo, baixo custo) 

\section{A evolução do circuito}
 
Valendo-se da propriedade logarítmica do termistor, uma pequena variação de temperatura ambiente ocasionada pelo processo de expiração ocasionaria uma mudança exponencial no valor da resistência. Por esse motivo, na origem do projeto, era esperada uma medição simples, obtida através da variação de tensão em um divisor resistivo composto por um termistor e uma resistência padrão(Figura: \ref{fig:divisorResistivo}). Contudo, devido alguns contratempos práticos, diversas mudanças fizeram-se necessárias. 
 

 
Uma das principais limitações para o desenvolvimento do projeto deve-se à dificuldade na aquisição dos componentes eletrônicos que, quando comprados via internet, necessitavam de um tempo para entrega relativamente alto e, para a compra realizada diretamente na loja física, existe uma limitação na variedade de lojas especializadas na cidade do Rio de Janeiro. Outra complicação relevante é referente à ausência de datasheet, que não é informado no momento da compra e torna-se impossível inferir qual é o datasheet correto apenas observando o componente, que é muito pequeno, sem qualquer informação sobre o fabricante ou o modelo. Dadas essas condições iniciais, foi construído o primeiro divisor resistivo apenas com base na informação de que o sensor adquirido tratava-se de um termistor NTC (do inglês Negative Temperature Coefficient) com resistência em temperatura ambiente de $10K\Omega$. Foi utilizada uma fonte de alimentação comercial de $12V$ e utilizado diversos valores entre $10K\Omega$ e $330\Omega$ para a resistência $R2$, contudo para nenhum valor de $R2$ era observada qualquer alteração de tensão na medida em que o ar era exalado próximo ao sensor, contrariando ao que era esperado, dado que a queda de tensão em cima do termistor deve ser variável junto à alteração na resistência.
 
\begin{figure}[h!]
	\begin{center}
 		\includegraphics[width=0.3\linewidth]{images/divisor_resistivo.png}
 		\caption{Divisor Resistivo}
 		\label{fig:divisorResistivo}
 	\end{center}
\end{figure}
 
A lei de resfriamento de newton (\ref{eq:resfriamento}) indica que a taxa com que um corpo perde calor é proporcional à diferença de temperatura entre o corpo e o meio  no qual ele se encontra. Valendo-se desse princípio, nota-se que uma baixa diferença de temperatura resultaria em um maior tempo de resposta do sensor. Portanto, para contornar o problema obtido, o circuito foi ajustado para utilizar a propriedade de autoaquecimento do termistor (\ref{eq:autoAquecimento}). 
 
  
\begin{equation} \label{eq:resfriamento}
	\dfrac{dQ}{dt} = h.A.(T(t) - T_{env}) = h.A \Delta T(t)	
\end{equation}
Onde:
\begin{itemize}[label=]
	\item $Q$: Energia térmica
	\item $t$: Tempo
	\item $h$: Coeficiente de transferência de calor
	\item $A$: Área de transferência de calor
	\item $T$: Temperatura do objeto
	\item $T_{env}$: Temperatura do ambiente
\end{itemize}
 
\begin{equation} \label{eq:autoAquecimento}
	T_0 = T(R) - \dfrac{V^2}{KR}
\end{equation}
Onde:
\begin{itemize}[label=]
	\item $T_0$: Temperatura do meio
 	\item $T(R)$: Temperatura do termistor em função de sua resistência
 	\item $V$: Diferença de potencial entre os terminais do termistor
 	\item $K$: Fator de dissipação do termistor
 	\item $R$: Resistência
\end{itemize}
 
 
 
Ao aplicar uma alta corrente no sensor, é induzido então um aumento na temperatura deste. Em decorrência desse aumento, é possível observar uma maior diferença de temperatura entre o sensor e o fluxo de ar e, por conseguinte, uma maior taxa para transferência de calor e um menor tempo de resposta por parte do termistor. Contudo, para atingir uma faixa de temperatura sensível, capaz de gerar uma resposta visível ao expor o sensor à respiração, o circuito necessita de uma tensão elevada, acima das entregues por fontes comerciais padrão, que costumam variar entre $5V$ e $12V$, gerando a necessidade de projetar um retificador de tensão capaz de converter a tensão de corrente alternada entregue pela rede elétrica residencial em uma tensão contínua alta o suficiente para fornecer ao termistor a corrente demandada.
 
Realizando testes de bancada, com um gerador de tensão variando de $0V$ à $30V$ e uma resistência de $330\Omega$ em série com o termistor, foi possível atingir uma temperatura sensível ao sopro, entretanto, ao aplicar uma tensão em torno de $22V$, o sensor demorava um tempo considerável para aquecer novamente, tornando-o inviável para medir o comportamento respiratório dada a frequência do sopro em uma respiração normal. Aumentando a tensão para $28V$, já era possível observar uma atenuação considerável na queda constante da temperatura dado que o termistor era capaz de se aquecer mais rápido. O problema gerado por esse aumento de tensão deve-se ao fato de que, quanto maior é a corrente, maior é o autoaquecimento e, quanto maior a temperatura, menor é a resistência, gerando um aumento ainda maior na corrente passante que, por sua vez, aumenta ainda mais a temperatura até que o sensor atingia um patamar no qual queimava, caso não houvesse nenhum sopro forçando a temperatura a diminuir. 
 
Se por um lado, o aumento da tensão era importante para que a temperatura não decaísse constantemente ao iniciar a medição respiratória, por outro, o sistema precisava ser protegido para que a corrente não atingisse um determinado patamar que danificasse o componente. O ajuste mais simples para regular a corrente em um componente costuma ser adicionar uma resistência em série, diminuindo a corrente por uma mera consequência da lei de ohm (\ref{eq:leideohm}). Contudo, por se tratar de uma resistência variável, a tensão necessária para o aquecimento do termistor em seu estado inicial era maior que a tensão necessária para mantê-lo abaixo de um patamar seguro após a diminuição de sua resistência pelo efeito do autoaquecimento. Foi pensado então em um circuito de controle não linear (Figura: \ref{fig:circuitoRealimentado}), que, em teoria, forneceria uma alta tensão para o termistor até que sua resistência variasse e a corrente passante atingisse um patamar determinado. 
  
\begin{equation} \label{eq:leideohm}
	V = R.I
\end{equation}
Onde:
\begin{itemize}[label=]
  	\item $V$: Tensão
  	\item $R$: Resistência
  	\item $I$: Corrente
\end{itemize}

\begin{figure}[h!]
	\begin{center}
		\includegraphics[width=1\linewidth]{images/Circuito_de_controle.png}
		\caption{Circuito de Controle}
		\label{fig:circuitoRealimentado}
	\end{center}
\end{figure}

O comportamento esperado para o circuito de controle seria o seguinte: O amplificador operacional iria ajustar a tensão de saída na tentativa de igualar as tensões nos dois terminais de entrada, sendo assim, no momento inicial, quando a tensão no terminal de entrada negativo é igual a zero e a do terminal positivo é igual a $5V$, o amplificador aumenta sua saída fazendo com que o transistor entre em saturação e conduza uma corrente no emissor praticamente igual à de referência no coletor, na medida em que a resistência do termistor diminui pelo efeito de autoaquecimento, a tensão de saída aumenta até que esta se iguale à do terminal positivo, ao atingir esse patamar, o transistor saí da zona de saturação e o sistema trabalha pra manter a tensão de saída constante. No momento em que o sensor entra em contato com o fluxo de ar, a resistência do termistor aumenta, diminuindo a tensão na saída e estimulando o sistema de controle a saturar o transistor e induzir o efeito do autoaquecimento no sensor. Após quantizar a tensão da saída através de um microcontrolador, foi possível traçar o gráfico da figura \ref{fig:RespostaCircuitoRealimentado}, no qual aparece nítida a resposta do sistema ao submeter o sensor a um fluxo de ar. Entretanto, ao expor o sensor a um período prolongado de exposição à respiração humana, foi possível observar que o efeito do resfriamento contínuo continuava a ser reproduzido (gráfico da figura \ref{fig:RespostaCircuitoRealimentadoLongoPrazo}).


\begin{figure}[h!]
	\begin{center}
		\includegraphics[width=1\linewidth]{images/RespostaCircuitoRealimentado.png}
		\caption{Resposta em curto prazo circuito realimentado (Tempo no eixo horizontal e valor de tensão quantizado no eixo vertical)}
		\label{fig:RespostaCircuitoRealimentado}
	\end{center}
\end{figure}

\begin{figure}[h!]
	\begin{center}
		\includegraphics[width=1\linewidth]{images/DecaimentoRespiracaoMascara.png}
		\caption{Resposta de longo prazo circuito realimentado (Tempo no eixo horizontal e valor de tensão quantizado no eixo vertical)}
		\label{fig:RespostaCircuitoRealimentadoLongoPrazo}
	\end{center}
\end{figure}


%\section{A fonte de tensão}
 
 
  
  \include{04_Materiais}

    \chapter{Conclusão}
  
  \section{Limitações}
  
  As principais limitações deste trabalho são:
  
  \subsection{Hardware}
  
  Em relação ao hardware do projeto, a principal limitação encontrada é referente ao resfriamento do sensor na utilização de longo prazo. A última versão desenvolvida do circuito neste trabalho ainda não possui um sistema de controle eficaz o suficiente para contornar esse problema, tornando-o inviável para a utilização prática na medição respiratória de pacientes em estudos científicos.
  
  \subsection{Software}
  
  A não conclusão de um hardware funcional acarretou em uma limitação por parte do software, que foi desenvolvido a partir de um sinal simulado, sem entrada integrada com o equipamento. Muito embora neste trabalho tenha sido desenvolvido também um software capaz de ler o sinal de entrada e armazenar em um arquivo .csv, o sistema não opera de forma integrada, sendo necessário que o usuário execute um programa para gerar o arquivo .csv e, em seguida, fazer o upload desse arquivo no software responsável pela realização das operações matemáticas. 

  \section{Trabalhos futuros}
  
  As próximas evoluções do trabalho consistem em desenvolver um sistema de controle capaz de manter a corrente no termistor constante, aumentando o fornecimento de tensão sempre que o sensor resfriar e aumentar sua resistência, diminuindo o fornecimento de tensão sempre que a temperatura do sensor estiver ultrapassando à necessária para a medição respiratória e obtendo, através do erro, uma medida de tensão que refletirá o comportamento respiratório. Uma vez que esse sistema funcione de forma funcional, realizar a integração entre o sistema, o leitor de entrada e o software de cálculos.
  
  \section{Considerações finais}
  
  O desenvolvimento de um equipamento de baixo custo, capaz de monitorar o comportamento respiratório de pacientes para estudos científicos resulta em uma contribuição significativa para o desenvolvimento da sociedade, uma vez que, diversos estudos científicos poderiam se beneficiar de tal artefato. Em especial, o laboratório de neuroimagem e psicofisiologia, que realiza diversos estudos sobre o comportamento humano, incluindo estudos com análise e medições de dados respiratórios, beneficiaria-se com o desenvolvimento de um equipamento capaz de realizar tais medições por um baixo custo.  
  Muito embora a última versão do hardware desenvolvida não tenha sido satisfatória para o uso prático, este trabalho contribui para o desenvolvimento futuro indicando diversos problemas encontrados, bem como as soluções utilizadas para contorná-los, além de traçar um caminho bem definido para a continuação de seu desenvolvimento.
  Como entrega concreta e funcional, o trabalho culminou em um software capaz de realizar operações matemáticas de forma fácil e gerar gráficos que poderão ser utilizados pelos pesquisadores em suas análises.
  A resposta à pergunta fundamental deste trabalho, referente à viabilidade do desenvolvimento de um equipamento de baixo custo capaz de realizar uma medição respiratória confiável, apesar de por hora ainda inconclusiva uma vez que o equipamento não foi, de fato, construído de maneira funcional, tende a ser positiva, avaliando os problemas e análises encontrados neste trabalho, bem como a sua projeção para trabalhos futuros.

  \backmatter
  \bibliographystyle{coppe-unsrt}
  \bibliography{example}

  \appendix
  \chapter{Software teste de conceitos}
  
  \begin{figure}[h!]
  	\begin{center}
  		\includegraphics[width=1\linewidth]{images/teste_de_conceitos1.pdf}
  	\end{center}
  \end{figure}
    \begin{figure}[h!]
  	\begin{center}
  		\includegraphics[width=1\linewidth]{images/teste_de_conceitos2.pdf}
  	\end{center}
  \end{figure}
  \begin{figure}[h!]
	\begin{center}
		\includegraphics[width=1\linewidth]{images/teste_de_conceitos3.pdf}
	\end{center}
\end{figure}
  \begin{figure}[h!]
	\begin{center}
		\includegraphics[width=1\linewidth]{images/teste_de_conceitos4.pdf}
	\end{center}
\end{figure}
\end{document}
%% 
%%
%% End of file `example.tex'.
