\chapter{Motivação}
  
\section{Fisiologia da Respiração} \label{sec:fisiologiadarespiracao}

	A respiração é, em resumo, um ato automático do qual o controle é exercido por centros respiratórios no tronco encefálico, onde são produzidos impulsos neuronais para os músculos respiratórios. 
	
	A inspiração possui como músculo principal o diafragma, que se contrai, desce e expande a cavidade torácica, comprimindo então o conteúdo abdominal e empurrando para fora a parede abdominal. Concorrentemente, os músculos da caixa torácica também expandem o tórax, em especial, os músculos escalenos que percorrem das vértebras cervicais às duas primeiras costelas e os músculos intercostais paraesternais, que possuem um trajeto desde o externo até as costelas. Na medida em que o tórax se expande, a pressão intratorácica diminui, deslocando o ar da árvore traqueobrônquica para os alvéolos, preenchendo os pulões em expansão. O oxigênio é então difundido para os capilares pulmonares adjacentes, enquanto o dióxido de carbono sai do sangue para os alvéolos.
	
	Quando ocorre a expiração, a parede torácica bem como os pulmões retraem-se, o diafragma relaxa, elevando-se passivamente, enquanto o ar flui para fora do corpo.
	
	Em estado normal, a respiração é tranquila, audível próximo à boca, sendo possível observar apenas os movimentos abdominais com facilidade. Contudo, durante a prática de atividade física, ou em decorrência de determinadas doenças, faz-se necessário um esforço respiratório adicional, recrutando músculos acessórios e demandando ainda mais esforço dos músculos abdominais, que passa a auxiliar também na expiração.
	
\section{Padrões Respiratórios} \label{sec:padroesrespiratorios}

	A faixa de frequência padrão para adultos normais é de, aproximadamente, 14 à 20 incursões por minuto e até 44 em lactantes. Em sua normalidade, a inspiração e expiração possuem o mesmo tempo e amplitude, sendo intercalados por uma leve pausa. No momento em que uma dessas características é modificada, surgem os ritmos respiratórios anormais, tais quais a respiração de Cheyne-Stokes, respiração de Biot, respiração de Kussmaul e respiração suspirosa.
	
	Respiração de Cheyne-Stokes: Frequentemente causada por insuficiência cardíaca, hipertensão intracraniana, acidentes vasculares cerebrais e traumatismos cranioencefálicos. Caracteriza-se por uma fase de apneia seguida de incursões inspiratórias cada vez mais profundas até atingir um máximo e, em seguida, decrescer até uma nova pausa. Esse comportamento ocorre devido a variações da tensão de $O_2$ e $CO_2$ no sangue. O excesso de $CO_2$ durante o período de apneia obriga os centros respiratórios a enviar estímulos mais intensos, resultando em um aumento da amplitude dos movimentos respiratórios, em consequência, haverá uma maior eliminação de $CO_2$ fazendo com que a concentração deste no sangue decaia. Após essa queda, os centros respiratórios recebem o comando inverso, enviando estímulos menores e diminuindo a amplitude dos movimentos respiratórios até que aumente novamente a concentração de $CO_2$ no sangue e o ciclo se repita.
	
	Respiração de Biot (ou atáxica): As causas dessa respiração são similares às da respiração de Cheyne-Stokes, porém, no ritmo de Biot, a respiração possui duas etapas, uma de apneia seguida de outra com movimentos respiratórios anárquicos quanto ao ritmo e à amplitude. Esse padrão respiratório quase sempre indica um grave comprometimento cerebral.
	
	Respiração de Kussmaul: Costuma ser causada pela acidose (diminuição do PH sanguíneo para menos de 7,35[aumento de H+]), principalmente a diabética. O padrão respiratório é composto de quatro fases: Inspirações ruidosas, gradativamente mais amplas, alternada com inspirações rápidas e de pequena amplitude; Apneia em inspiração; Expirações ruidosas, gradativamente mais profundas, alternadas com inspirações rápidas e de pequena amplitude;	
	E apneia em expiração.
	
	Respiração suspirosa: Normalmente, pode ser traduzida em tensão emocional e ansiedade. Seu padrão ocorre quando o paciente executa uma série de movimentos inspiratórios de amplitude crescente, seguidos de uma expiração breve e rápida.
	
	Além desses padrões respiratórios, existem diversos outros padrões observados (como a Taqpineia, Bradpneia, Respiração obstrutiva e Hiperpneia) que possuem correlação com patologias ou alterações emocionais e, portanto, existe um vasto campo de estudo acerca desse tema.
	
\section{Aplicação do equipamento}

	Motivado pela parceria com o "Laboratório de Neuroimagem e Psicofisiologia", situado no Hospital Universitário Clementino Fraga Filho, ocorre o desenvolvimento deste projeto. Orientado principalmente a estudos na área de neurociência comportamental, o desenvolvimento do equipamento visa trazer mais insumos às análises realizadas nos diversos projetos existentes no laboratório. Conforme mencionado na sessão \ref{sec:padroesrespiratorios}, existem diferentes padrões respiratórios já observados entre os quais se é possível obter uma relação de correlação com patologias e alterações emocionais e comportamentais. Por derradeiro, a existência de um equipamento capaz de obter dados referentes ao comportamento respiratório transcreve-se em um ganho à neurociência comportamental pelo qual este projeto se justifica.
